%% start of file `template.tex'.
%% Copyright 2006-2013 Xavier Danaux (xdanaux@gmail.com).
%
% This work may be distributed and/or modified under the
% conditions of the LaTeX Project Public License version 1.3c,
% available at http://www.latex-project.org/lppl/.

\documentclass[11pt,letterpaper,roman]{moderncv}        % possible options include font size ('10pt', '11pt' and '12pt'), paper size ('a4paper', 'letterpaper', 'a5paper', 'legalpaper', 'executivepaper' and 'landscape') and font family ('sans' and 'roman')

% moderncv themes
\moderncvstyle{banking}                             % style options are 'casual' (default), 'classic', 'oldstyle' and 'banking'
\moderncvcolor{blue}                               % color options 'blue' (default), 'orange', 'green', 'red', 'purple', 'grey' and 'black'

% adjust the page margins
\usepackage[scale=0.75]{geometry}

% hyperlink styles
\AfterPreamble{
\hypersetup{
  colorlinks=true
  urlcolor=magenta
}
}

% personal data
\name{D. Hudson}{Smith}
\title{\Large{curriculum vitae}} 
%\phone[home]{864.423.4713}
\email{dane2@clemson.edu}                               % optional, remove / comment the line if not wanted
\social[linkedin]{dhudsmith}                        % optional, remove / comment the line if not wanted
\social[github]{dhudsmith}                              % optional, remove / comment the line if not wanted

\usepackage[backend=biber,style=nature, sorting=none, maxbibnames=99]{biblatex} % bibliography
\usepackage[american]{babel} % helps convert TeX to American English
\usepackage{tabto} % spacing package

% add citation resources
\addbibresource{pubs_ml.bib}
\addbibresource{pubs_phys.bib}
\addbibresource{pres_ml.bib}
\addbibresource{pres_phys.bib}

\sethintscolumnlength{1.2in}
%----------------------------------------------------------------------------------
%            content
%----------------------------------------------------------------------------------
\begin{document}

\makecvtitle

\section{Research Vision}
I develop techniques for incorporating prior knowledge into machine learning systems for data-constrained applications. This synthesis of established domain knowledge with flexible machine learning methods leads to better anomaly detection rates, improved sample efficiency, and explainable inference, without sacrificing the expressive power of data-driven techniques like deep learning.

\section{Positions}
\cventry{1/2023 -- present}{Research Computing and Data, CCIT}{Director of Applied Machine Learning}{Clemson, SC}{}{}
\cventry{7/2022 -- present}{Clemson/MUSC Artificial Intelligence Hub}{Co-Lead}{Clemson, SC}{}{}
\cventry{10/2019 -- present}{Holcombe Dept. of ECE}{Research Assistant Professor}{Clemson, SC}{}{}  % arguments 3 to 6 can be left empty
\cventry{3/2018 -- 12/2022}{Watt Family Innovation Center}{Machine Learning Research Associate}{Clemson, SC}{}{}
\cventry{12/2016 -- 3/2018}{Dynamit Technologies}{Data Scientist}{Columbus, OH}{}{}

\section{Education}
\cventry{2011-2016}{Ph.D. in Theoretical Atomic Physics}{Ohio State University}{Columbus, OH}{GPA: 3.9/4.0}{
  Research: theoretical atomic physics, quantum field theory, computer simulation of quantum systems
  \newline{}Thesis: \href{https://etd.ohiolink.edu/apexprod/rws_etd/send_file/send?accession=osu1478192866433031&disposition=inline}{Inducing Resonant Interactions in Ultracold Atoms with an Oscillating Magnetic Field}
  \newline{}Advisor: Dr. Eric Braaten, \url{braaten.1@osu.edu}
}

\cventry{2007-2011}{B. S. in Physics, B. A. in Mathematics}{Erskine College}{Due West, SC}{GPA: 3.9/4.0}{}

\section{Publications}
\subsection{Machine learning and Artificial Intelligence}

\begin{refsection}[pubs_ml.bib]
  \nocite{*}
  \printbibliography[heading=none]
\end{refsection}

\subsection{Atomic Physics}

\begin{refsection}[pubs_phys.bib]
  \nocite{*}
  \printbibliography[heading=none]
\end{refsection}

\section{Research awards}

\cvitemwithcomment{1/2024}{NIH COBRE Pilot Award}{\$350k}
\quad (Co-PI) Team Science Supplement: SC COBRE for Translational Research Improving Musculoskeletal Health (SC TRIMH)

\cvitemwithcomment{9/2023}{CDC Center for Forecasting and Outbreak Analytics}{\$7MM}
\quad (Senior Personnel) Disease Modeling and Analytics to inform Outbreak Preparedness, Response, Intervention, Mitigation, and Elimination in South Carolina (DMA-PRIME)

\cvitemwithcomment{8/2023}{NIH COBRE Renewal Award}{\$11MM}
\quad (Co-PI) COBRE Renewal: Translational Research Improving Musculoskeletal Health (TRIMH) (SC TRIMH)

\cvitemwithcomment{7/2023}{Clemson/MUSC AI Hub Augmentation Award}{\$25k}
\quad (Co-PI) Automated Identification of Patent Ductus Arteriosus using a Computer Vision Model

\cvitemwithcomment{4/2023}{Clemson University Research Foundation Technology Maturation Award}{\$20k}
\quad (Senior Personnel) Integration of Smart Needle Measurements and Functional Ultrasonography by Machine Learning for Quantitative Monitoring and Assessment of Acupuncture Myofascial Pain Management

\cvitemwithcomment{1/2022}{Clemson-MUSC Artificial Intelligence Hub}{\$5k}
\quad (Fellow) Artificial Intelligence Advocate

\cvitemwithcomment{7/2021}{RHBSSI Seed Grant (Clemson University)}{\$45k}
\quad (Co-PI) ColorNet: Developing AI-based color correction tools for sports media applications

\cvitemwithcomment{6/2021}{ACRE Competitive Grants Program (SCDA)}{\$120k}
\quad (PI) AI Master Gardener for Greenhouse Supervision

\cvitemwithcomment{4/2021}{Prisma Health Seed Grant}{\$20k}
\quad (Senior Personnel) Automated Quality Assessment of FAST Exams

\cvitemwithcomment{2/2021}{ACRE Competitive Grants Program (SCDA)}{\$20k}
\quad (Co-PI) AI for Fruit and Vegetable Harvesting in South Carolina

\cvitemwithcomment{2/2021}{CU Seed Grant, Tier 1 (Clemson University)}{\$5k}
\quad (Co-PI) ColorNet: An AI-based color management system for live video

\cvitemwithcomment{11/2019}{CURF Tech Maturation Fund (Clemson)}{\$29k}
\quad (Co-PI) ColorNet: Consistent display of Clemson brand colors using artificial intelligence

\cvitemwithcomment{8/2019}{Erwin Center for Brand Communications (Clemson University)}{\$8k}
\quad (Co-PI) AI for on the fly color correction of sports footage

\cvitemwithcomment{7/2018}{ACRE Competitive Grants Program (SCDA)}{\$105k}
\quad (Co-PI) Rapid Chicken Sex Determination with Multiple Mechanisms and AI


\section{Teaching Experience}
\subsection{Mentorship Experience}
\cvitem{Graduate students}{I co-advise one PhD student in Biomedical Data Science and Informatics, two PhD students in the School of Computing, and one DMD student in the Department of Orthodontics at MUSC.}
\cvitem{Undergraduates}{In my 5 years leading the Watt AI program, I worked with over 100 Creative Inquiry students applying Machine Learning techniques to research projects at Clemson.}
\cvitem{Interns}{Since 2018, I have mentored more than 15 undergraduate student interns. These interns worked on machine-learning related projects in diverse disciplines. }
\subsection{Clemson University}
\cvitem{Spring 2023--present}{Instructor for Data Science, Machine Learning,
  and Deep Learning Workshops for Research Computing and Data. Selected topics
  include "Data Visualization in Python", "Introduction to Deep Learning with Pytorch",
  and "Attention, Transformers, and LLMs: a hands-on introduction in Pytorch"}
\cvitem{Spring 2018--Fall 2022}{Instructor for Watt AI Creative Inquiry course for 9 consecutive semesters}
\cvitem{Spring 2020--Fall 2021}{Designed intro to artificial intelligence curriculum for undergraduates from diverse majors}
\cvitem{Fall 2021--Fall 2022}{Led weekly journal club with advanced students}
\cvitem{Fall 2018--Spring 2019}{Instructor for Ulbrich CI focused on manufacturing analytics}

\subsection{Ohio State University}
\cvitem{Fall 2015}{Tutor for graduate level classical mechanics course}
\cvitem{Fall 2012--Spring 2013}{Recitation and lab instructor for Physics: Vibrations, Fluids, Thermodynamics, and Special Relativity}

\subsection{Erskine College}
\cvitem{Spring 2010}{Lab instructor for Modern Physics}
\cvitem{Fall 2009}{Teaching assistant for Calculus}
\cvitem{Fall 2008--Fall 2009}{Teaching assistant for Introductory Physics}
\cvitem{Fall 2008--Fall 2010}{Writing assistant for various subjects}

\section{Presentations}
\subsection{Machine learning}

\begin{refsection}[pres_ml.bib]
  \nocite{*}
  \printbibliography[heading=none]
\end{refsection}

\subsection{Physics}
\begin{refsection}[pres_phys.bib]
  \nocite{*}
  \printbibliography[heading=none]
\end{refsection}

\section{Computational tools}
\begin{itemize}
  \item Python, R, SQL, bash, C++, C\#, Java, LaTeX
  \item Deep Learning and Probabilistic Programming: Pytorch, Pyro, NumPyro, Jax
  \item Experience with Cloud and cluster computing environments
  \item Hardware-accelerated array programming for scientific computing
\end{itemize}

\section{Honors and Awards}
\cvline{2016}{Presidential Fellow, OSU}
\cvline{2013}{Winner of Physics Dept. Poster Competition, OSU}
\cvline{2011}{Fowler Fellow, OSU}
\cvline{2011}{University Fellow, OSU}
\cvline{2010}{T. Kincannon Mathematics Award, Erskine College}
\cvline{2010}{Junkin Physics Award, Erskine College}
\cvline{2008}{Garnet Circle Award, Erskine College}
\cvline{2007}{Roy M. Smith Mathematics Scholarship, Erskine College}


\clearpage
\end{document}


%% end of file `cv.tex'.

%%% Local Variables:
%%% mode: latex
%%% TeX-master: t
%%% End:
